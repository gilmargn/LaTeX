\documentclass[12pt]{exam}
\usepackage[utf8]{inputenc}
\usepackage[portuguese,brazil]{babel}
\usepackage[margin=.65in]{geometry}
\usepackage{amsmath,amssymb}
\usepackage{multicol}
\usepackage{multirow}
\usepackage{float}
\usepackage{enumerate}
\usepackage{enumitem}
\usepackage{array}
\usepackage{graphicx}% delete the demo option in your actual code
\usepackage{listings}
\usepackage{circuitikz}    
\usepackage{xcolor}
\usepackage{tikz}
\usetikzlibrary{shapes.multipart, calc}
\usepackage{rotating}
\usepackage{caption}
%\usepackage[version=3]{mhchem}

\pointpoints{ponto}{pontos}
\hpword{Pontos:}
\vpword{Pontos}
\htword{Total}
\vtword{Total:}
\vsword{Resultado}

\definecolor{azul}{RGB}{0,32,96}
\definecolor{verde}{RGB}{0,132,124}
\definecolor{cinza}{RGB}{0,84,132}
%\definecolor{cinza2}{RGB}{218,220,216}
\definecolor{cinza2}{RGB}{132,134,136}
\definecolor{mygreen}{rgb}{0,0.6,0}
\definecolor{mygray}{rgb}{0.5,0.5,0.5}
\definecolor{mymauve}{rgb}{0.58,0,0.82}
\newcommand*{\vttfamily}{%
  \fontencoding{T1}\fontfamily{lmvtt}\selectfont
}
\lstdefinestyle{Cstyle}{%
language=C,numbers=left, keywordstyle=\color{mymauve}\vttfamily,commentstyle=\color{mygreen}\vttfamily,stringstyle=\color{blue},basicstyle=\scriptsize\vttfamily,numberstyle=\tiny\vttfamily,breaklines=true,stepnumber=1,tabsize=2,xleftmargin=\leftmarginii,escapechar=!,captionpos=b,morekeywords={include, printf,NULL},columns=flexible,escapeinside={(^}{^)},
frame=l,
framesep=2.5mm,
framexleftmargin=2.5mm,
fillcolor=\color{cinza2!50},
rulecolor=\color{ballblue},
rulecolor=\color{blue}
}

\lstnewenvironment{C}[1][]
{
\lstset{style=Cstyle,#1}
}
{}


\newcommand{\class}{Informática Básica}
\newcommand{\profname}{Gilmar Gomes do Nascimento}
\newcommand{\examnum}{Avaliação} \newcommand{\examdate}{$19/06/2023$}
\newcommand*{\oldneg}{\mathord{\nao}}
\pagestyle{headandfoot}
\runningheadrule
\firstpageheader{}{}{}
\runningheader{\class}{\examnum}{\examdate}
\runningfootrule
\firstpagefooter{}{}{}
\runningfooter{}{Página \thepage\ de \numpages}{}
%\pagestyle{foot}
%\lfoot{}
%\cfoot[]{Página \thepage\ de \numpages}
%\rfoot{}
%
%\runningfootrule
%\boxedpoints %\pointsinmargin
%\printanswers
%\SolutionEmphasis{\color{red}}
%\definecolor{SolutionBoxColor}{gray}{0.8}
%\renewcommand{\solutiontitle}{\noindent\textbf{Solução:}\enspace}
%\renewcommand{\solutiontitle}{\noindent\textbf{Solution:}\par\noindent}

\begin{document}

\noindent
\begin{tabular*}{\textwidth}{l @{\extracolsep{\fill}} r @{\extracolsep{6pt}} l}
\textbf{Disciplina:} \class &  \multicolumn{2}{c}{\multirow{4}{*}{
\includegraphics[scale=0.5]{logo.png}}}\\
\textbf{Prof.:} ~\profname \vspace{0.2in} && \\
\vspace{0.2in} \textbf{Nome:} \makebox[3in]{\hrulefill} \vspace{0.2in} &&\\
%\textbf{Matrícula:} \makebox[1.5in]{\hrulefill} \textbf{Nota:} \makebox[1in]{\hrulefill} &&\\ 
\multicolumn{3}{c}{\textbf{{\examnum \hspace{3px} \examdate}}}\\
\multicolumn{3}{c}{}\\
\end{tabular*}\\
\rule[2ex]{\textwidth}{1.6pt}
\parskip=0.1in

%Este exame contêm \numpages\ páginas (incluindo esta capa) e \numquestions\ questões.\\
%Total de \numpoints~ pontos.

%\begin{center}
%Grade Table (for teacher use only)\\
%\addpoints
%\gradetable[v][questions]
%\end{center}

\noindent
%\input{shark}
\begin{questions}
\question Sobre a história dos computadores. Marque V ou F para cada assertiva\\
(\hspace{6px}) Santos Dumont é o nome do primeiro computador projetado e construído no Brasil. \\
(\hspace{6px}) Patinho feio é o nome do supercomputador brasileiro. \\
(\hspace{6px}) Alan Turing é considerado o pai da computação. \\
(\hspace{6px}) Condessa de Lovelace é reconhecida principalmente por ter escrito o primeiro algoritmo para ser processado por uma máquina.\\
(\hspace{6px}) O primeiro computador pessoal é o Apple I. \\
(\hspace{6px}) ENIAC (Electronic Numerical Integrator and Computer, ou em português, Computador Integrador Numérico Eletrônico) foi o primeiro computador eletrônico da história. 


\question Sobre mouse. Marque V ou F para cada assertiva\\
(\hspace{6px}) Existe mouse mecânico e é conhecido como mouse com bolinha \\
(\hspace{6px}) O projeto inicial do mouse foi desenvolvido na Xerox. \\
(\hspace{6px}) O mouse localizado entre as teclas se chama \textit{Pointstick} \\
(\hspace{6px}) O touchpad pode ser capacitivo e resistivo e é encontrado principalmente em \textit{laptops/notebooks}
(\hspace{6px}) O \textit{trackball} é um exemplo de mouse ergonômico. \\
(\hspace{6px}) \textit{Mousepad} também é conhecido como almofada para usar o mouse


\question Sobre teclados. Marque V ou F para cada assertiva\\
(\hspace{6px}) Existe um leiaute de teclado conhecida como Dvorak, porém é menos ergonômico\\
(\hspace{6px}) É possível adicionar outros leiautes virtuais de teclado. \\
(\hspace{6px}) O teclado QWERTY só existe em dispositivos móveis. \\
(\hspace{6px}) Existem teclados mecânicos, contudo eles usam uma membrana de silicone. \\
(\hspace{6px}) Teclas de controle é composto pelas teclas F1 até a F12,ESC e outras teclas da parte superior. \\
(\hspace{6px}) Teclas de função é composto pelas teclas: Tab, Shift, CTRL, ALT, ALT FN, Backspace e afins. \\
(\hspace{6px}) Três luzes se acendem no teclado: Capslock(Fixa), Numlock e scroll lock\\
(\hspace{6px}) Um teclado retroiluminado pode ser ativado ou desativado por uma tecla ou um \textit{driver}. \\
(\hspace{6px}) As teclas F, J e 5 possuem um relevo para orientar quem digita sem olhar para o teclado.\\
(\hspace{6px}) CTRL ESC combinado tem a mesma função que a tecla Super(Windows) que é mostrar o menu Iniciar.
\question Sobre alguns atalhos com o teclado. Marque V ou F para cada assertiva\\
(\hspace{6px}) CTRL Home leva o cursor sempre para o início do texto\\
(\hspace{6px}) Shift e setas de direção seleciona texto. \\
(\hspace{6px}) ALT w exibe a interrogação \\
(\hspace{6px}) Tecla Super(Windows) + L bloqueia a tela \\
(\hspace{6px}) ALT TAB alterna entre janelas. \\
(\hspace{6px}) ALT barra de espaço ativa as opções para minimizar, maximar, fechar e mover janela. \\
(\hspace{6px}) Tecla Super(Windows) + Seta de direção ``move" a janela conforme a direção informada. 

\question Sobre as apresentações. Marque V ou F para cada assertiva\\
(\hspace{6px}) Desenho vetorial é o uso de primitivas geométricas como pontos, linhas, curvas e formas ou polígonos - todos os quais são baseados em expressões matemáticas - para representar imagens em computação gráfica. \\
(\hspace{6px}) CMYK é mais usado em impressoras e na técnica de \textit{offset} \\
(\hspace{6px}) RGB é o padrão de monitores e projetos virtuais. \\
(\hspace{6px}) Dados raster, matriciais ou bitmap (que significa mapa de bits em inglês) são imagens que contêm a descrição de cada pixel.\\
(\hspace{6px}) São exemplos de navegadores: Firefox, Chrome, Edge\\
(\hspace{6px}) São exemplos de buscadores: Edge, Bing, Google, Duck Duck Go\\
(\hspace{6px}) GPL é uma licença de códigos. \\
(\hspace{6px}) O Projeto GNU surgiu antes do projeto Linux\\
(\hspace{6px}) São exemplos de editores de imagem: CorelDraw, Photoshop, GIMP.\\
(\hspace{6px}) São exemplos de editores de vídeo: Vegas, Adobre Premiere, Pitivi.\\

\question Sobre o Microsoft Power Point. Marque V ou F para cada assertiva\\
(\hspace{6px}) O atalho para criar um novo slide é CTRL M. \\
(\hspace{6px}) O atalho para criar um novo arquivo é CTRL O. \\
(\hspace{6px}) Iniciar uma apresentação do slide atual basta usar SHIFT F5\\
(\hspace{6px}) A extensão dos arquivos em PowerPoint é pptx. \\
(\hspace{6px}) Não é possível inserir imagens, vídeos ou links.\\


\question Sobre o Microsoft Excel. Marque V ou F para cada assertiva \\
(\hspace{6px}) Para preencher células com dias da semana ou números basta usar arrastar a alça numa célula selecionada.\\
(\hspace{6px}) Para alterar o conteúdo de uma célula, basta apertar a tecla F3\\
(\hspace{6px}) Pasta de trabalho é o nome que se dá ao conjunto de planilhas \\
(\hspace{6px}) As linhas são representadas por letras, enquanto as colunas por números\\
(\hspace{6px}) Mesclar células é aglomerar várias células transformando em uma.\\
(\hspace{6px}) A fórmula cont.se serve para contar quantas vezes um conteúdo aparece.\\
(\hspace{6px}) O símbolo $\Sigma $ é utilizado para automatizar a soma\\
(\hspace{6px}) Para inserir uma fórmula ou operação matemática é necessário inserir o símbolo = \\

\question Sobre o Microsoft Windows. Marque V ou F para cada assertiva \\
(\hspace{6px}) Na calculadora do Windows não é possível usar a opção Programador\\
(\hspace{6px}) Pode ser adicionado mais de um horário no opções data e hora\\
(\hspace{6px}) Para mostrar a área de trabalho, basta usar a combinação Windows A \\
(\hspace{6px}) Não é possível criar pastas com o nome con\\
(\hspace{6px}) O Windows permite a criação de várias contas/usuários\\
(\hspace{6px}) Para capturar a tela o atalho é Windows Shift S.\\
(\hspace{6px}) O Windows não possui um programa no estilo \textit{bash}\\
(\hspace{6px}) Os botões para fechar a janela ficam à esquerda.\\
(\hspace{6px}) EXT4 é um sistema de arquivos que gerencia o acesso a arquivos em HDs e outras mídias mais comum e recomendável para o uso em dispositivos.  \\
(\hspace{6px}) Shift Delete apaga o arquivo sem passar pela lixeira.

\end{questions}
%\rule[2ex]{\textwidth}{2pt}
%\input{shark}
%\input{minuta}
%\input{ferramentas}
%\input{expressoes}
%\input{binarios}
%\input{shark.tex}
%\begin{questions}

\question O que são objetos no paradigma orientado a objetos?
\begin{solutionbox}{3.2in} 		
	\end{solutionbox}
\question O que são classes? Detalhe.

	\begin{solutionbox}{3.2in}
		
	\end{solutionbox}
\question O que são métodos? Descreva dois métodos.


	\begin{solutionbox}{2.3in}
		
	\end{solutionbox}

\question O que é herança no paradigma orientado a objetos?

	\begin{solutionbox}{2.3in} 		
	\end{solutionbox}
\question O que é polimorfismo no paradigma orientado a objetos?
	\begin{solutionbox}{2.3in} 		
	\end{solutionbox}
%\question 

\end{questions}
\end{document}
